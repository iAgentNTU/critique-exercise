\documentclass[a4paper]{article}
% Preamble Area

\usepackage{geometry} 
\usepackage{graphicx}


\newtheorem{assumption}{Assumption}

 
\title{Paper Commentary Exercise}
\author{Hungchi}

\begin{document}
\maketitle
% -----------------------
% Paper Name -- Rating: __/5
% (commentary here)
% 1 summary
% 3 positive topics, 1 criticism 
\section{-- Rating: 4/5}
\\

% positive #1: 3 key observations
這篇 Paper 是一份完整的 User case study。 之前此團隊發表過一篇 TAP & PLAY,是以介紹用數位筆錄音標記的 Toolkit,比較偏工具類,而這篇就是完整的將技術應用於 105 歲的老奶奶以及她的家庭中,因此整篇的敘述都是以質性的訪談調查為主。除此之外,也有針對錄音的行為進行分析與討論,觀察在不同的 iteration 裡,家庭成員錄音的情形。\\

% positive #2:
這篇用到的技術其實更貼近生活,比起用平板裝置,數位筆搭配傳統相簿是會帶給使用者更傳統的感受。技術上是在紙上設計特別的點陣圖,讓數位筆可以辨識不同區塊所代表的功能,達到錄音或播放等功能。同時可以在紙本上加上更多手寫的註解,例如:TAP HERE!! 讓使用行為非常貼近過去的習慣,對於長輩而言或許接受度更高。\\

% positive #3:
研究的範圍廣泛且完整,從系統的介入,使用行為的觀察到錄音記錄的分類與分析,最後甚至提到對於認知功能、社交以及身體功能的影響,還有對於照護者的支援。論文的最後再以對於未來其他相關研究要做老人語音相簿的推薦 Guildline,相當值得一讀。\\

% criticism:
我這邊單就技術面進行討論,傳統的紙本加上數位筆讓長輩更親切,但是某種程度上也限制了科技介入的空間。例如錄音筆收到的資料該如何反覆使用,系統介面如何更有彈性的變化都是傳統紙本比較難實現。這可以討論到傳統紙本跟數位的差異,數位化的優點就是易於搜尋、分享以及容易反覆提取利用。要怎麼在實體跟數位之間達到平衡,就是我的研究需要著墨的地方。


% ----------------------- import bibtex ----------------------------
\bibliographystyle{abbrv}
\bibliography{hungchi_critique}


\end{document}
