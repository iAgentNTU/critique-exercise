\documentclass[a4paper]{article}
% Preamble Area

\usepackage{geometry} 
\usepackage{graphicx}


\newtheorem{assumption}{Assumption}

 
\title{Paper Commentary Exercise}
\author{Janet Huang}

\begin{document}
\maketitle
% -----------------------
% Paper Name -- Rating: __/5
% (commentary here)
% 1 summary
% 3 positive topics, 1 criticism 
\section{Scalable Multi-label Annotation -- Rating: 4/5}
Deng \emph{et~al.} \cite{Deng2014} address the scalablity issue of multi-label annotation. They proposes an algorithm that exploits correlation, hierarchy, and sparsity of the label distribution to yield significant savings for image labeling tasks.\\

% positive #1: 3 key observations
The paper clearly identifies three key obersvations for labels in real world scenarios. The labels are correlated, sparse and naturally form a hierarchy.\\

% positive #2:
The authors propose a theoretical analysis and a practical algorithm to satify the assumptions. They clearly define the utlity and cost function and use a pseudo code to describe the algorithm.\\

% positive #3:
The paper applys the algorithm on a real image dataset in real world scenarios. The results show that their algorithm accuires up to 6× savings compared to the na\"{i}ve approach.\\

% criticism:
I wish they can provide a complete example to describe the process of anaylysis and demostrate the power of algorithm for reducing the costs. To achieve the goal of scalablity, we should select the high-utility queries with low cost. I wish the authors can provide their ideas on the trade-off between utility and cost.


% ----------------------- import bibtex ----------------------------
\bibliographystyle{abbrv}
\bibliography{janetyc_critique}


\end{document}
