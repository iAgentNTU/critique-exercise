\documentclass[a4paper]{article}
% Preamble Area

\usepackage{geometry} 
\usepackage{graphicx}


\newtheorem{assumption}{Assumption}

 
\title{Paper Commentary Exercise}
\author{Meng-Ying Chan}

\begin{document}
\maketitle
% -----------------------
% Paper Name -- Rating: __/5
% (commentary here)
% 1 summary
% 3 positive topics, 1 criticism 

\section{Wearable audio-feedback system for gait rehabilitation in subjects with parkinson?s disease -- Rating: 4/5}

% positive #1: 3 key observations
This paper \cite{Casamassima:2013:WAS:2494091.2494178} is clearly to present a system for gait rehabilitation for use in a daily life setting. \\

% positive #2:
This paper applies the algorithm on ABF(audio bio-feedback) application.\\

% positive #3:
The authors implements the system that entails the improvement of gait's rhythm amplitude, the decrease of the asymmetry and the correction of a wrong posture assumption.\\

% criticism:
I wish they can provide more complete example about the training experiment. 


\section{Physiotherapy management of knee osteoarthritis \cite{APL:APL1612} -- Rating: 4/5}


% positive #1: 3 key observations
The paper points that there is sufficient evidence to indicate that physiotherapy interventions can reduce pain and improve function in those with knee OA .\\

% positive #2:
This paper introduces some types to maintain our knees,including exercise, taping, bracing, insoles and shoes and manual therapy.\\ 

% positive #3:
This paper's reference give me lots of clues to find out the exercise that suit for OA.\\

% criticism:
I wish the paper can show more ways about the physical therapy.



% ----------------------- import bibtex ----------------------------
\bibliographystyle{abbrv}
\bibliography{meng_critique}


\end{document}
