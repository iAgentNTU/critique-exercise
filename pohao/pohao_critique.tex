\documentclass[a4paper]{article}
% Preamble Area

\usepackage{geometry} 
\usepackage{graphicx}


\newtheorem{assumption}{Assumption}

 
\title{Paper Commentary Exercise}
\author{PoHao CHOU}

\begin{document}
\maketitle
% -----------------------
% Paper Name -- Rating: __/5
% (commentary here)
% 1 summary
% 3 positive topics, 1 criticism 
\section{A Bayesian Method to Incorporate Background Knowledge during Automatic Text Summarization -- Rating: 4/5}
Louis \cite{P14-2055} proposed another aspect in summarization, which focuses on extracting new text deviating from previous knowledge on the topic. They proposed an algorithm that exploits Bayesian and KL-divergence to extract new and important sentences.\\

% positive #1: 3 key observations
The paper proposed an interesting issue on summarization. Rather than extracting general importances for each document, extracting novel information is more helpful in some applications.\\

% positive #2:
This inspired me that summarization can be variable, here the summarization is varied with knowledge corpus. Maybe there are others, like time or weather.\\ 

% positive #3:
The setting of experiment and dataset is clear, but I wish they can provide the comparison between small and big background corpus for topic-based and surprise-based method.\\

% criticism:
I wish the paper can show the reason or comparison to use Dirichlet distribution to model hypothesis probability.


% ----------------------- import bibtex ----------------------------
\bibliographystyle{abbrv}
\bibliography{pohao_critique}


\end{document}
