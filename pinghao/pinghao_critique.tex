\documentclass[a4paper]{article}
% Preamble Area

\usepackage{geometry} 
\usepackage{graphicx}


\newtheorem{assumption}{Assumption}

 
\title{Paper Commentary Exercise}

\author{Pinghao}


\begin{document}
\maketitle
% -----------------------
% Paper Name -- Rating: __/5
% (commentary here)
% 1 summary
% 3 positive topics, 1 criticism 

\section{Simultaneous Tracking & Activity Recognition (STAR)
Using Many Anonymous, Binary Sensors}

%subtitle:
Pros:\\

% positive #1:
1. Use only simple sensors like motion, pressure detectors instead of complicated, expensive ones to get a result.\\

% positive #2:
2. This work is highly applicapable and has great commercial value.\\

% positive #3:
3. It uses only filters with little training but does well in  practice.\\ 

% subtitle:
Cons:\\

% criticism:
It can not classify complicated activities.\\

\section{Online activity recognition using evolving classifiers}

%subtitle:
Pros:\\

% positive #1:
1. Using methods that works well on activity recognition in a growing data.\\

% positive #2:
2. Inherit many previous works trying on huge various of methods.\\

% positive #3:
3. Using only binary sensors to do on this work.\\ 

% subtitle:
Cons:
\\

% criticism:
Using an unbalanced dataset.\\

\section{Human Activity Recognition from Basic Actions Using
Finite State Machine}

%subtitle:
Pros:\\

% positive #1:
1. Using method that differs from probabilistic models and pattern recognitions to get good performance on activity recognition.\\

% positive #2:
2. Quite a simple way to do on the task.\\

% positive #3:
3. Transform the task from detecting activities to sub-activities.\\ 

% subtitle:
Cons:
\\

% criticism:
The author doesn't explain well how to get sub-activities from the video data.\\

% ----------------------- import bibtex ----------------------------


\end{document}
